\chapter{Literature Review and Related Work}
\label{chap:relatedworks}

To evaluate the current landscape of English learning platforms and games, we conducted a feature comparison of multiple solutions, including traditional platforms (Preply, British Council), mobile applications (Duolingo, ELSA, Anki), and gamified learning experiences (Bookworm Adventures, Scribblenauts Unlimited). The analysis focuses on key features such as personalized learning, vocabulary building, grammar instruction, and user engagement strategies.

\vspace{1em}
\noindent\textbf{Feature Comparison of Existing Solutions}

\vspace{0.5em}

A wide range of language learning solutions exists, each catering to different learning needs and styles. Traditional platforms such as Preply and the British Council emphasize structured learning approaches with human tutoring and well-defined lesson plans. However, these platforms may lack elements of interactivity and engagement found in newer solutions.

\vspace{1em}

Mobile applications such as Duolingo and ELSA leverage AI and gamification to enhance learning retention. Duolingo, for example, employs a streak-based reward system, bite-sized lessons, and adaptive exercises, making language learning more accessible and engaging. ELSA, on the other hand, specializes in pronunciation practice, utilizing AI-powered speech recognition to provide immediate feedback.

\vspace{1em}

Gamified experiences like Bookworm Adventures and Scribblenauts Unlimited offer an alternative approach by embedding language learning within an interactive storyline. These games encourage incidental learning through gameplay mechanics, allowing users to engage with language in a natural and enjoyable way. However, they often lack structured lessons and assessments, which may limit their effectiveness as primary learning tools.

\vspace{1em}

\noindent\textbf{Feature Comparison Table}

\vspace{1em}

The table below summarizes the key features of various English learning platforms:

\vspace{1em}


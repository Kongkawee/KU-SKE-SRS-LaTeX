\chapter{Requirement Analysis}
\label{chap:requirement-analysis}

\section{Stakeholder Analysis}
\label{section:stakeholder-analysis}

Stakeholders are individuals, groups, or entities that have an interest in or are affected by the development and implementation of \textit{Spell Splash}. Identifying stakeholders is essential to ensure that the game meets user expectations, aligns with educational objectives, and provides value to all involved parties. Below are the key stakeholders for \textit{Spell Splash}:

\vspace{1em}

\begin{enumerate}
    \item \textbf{Students (Aged 10 to 25)}\\
    The primary users of the game, aiming to enhance their English vocabulary through an engaging and interactive experience. 
    They are interested in a fun and effective learning method that differs from traditional memorization techniques. 
    Their learning experience and feedback will directly impact the game's effectiveness in achieving its educational objectives, ensuring that it remains engaging and beneficial for vocabulary retention.

    \item \textbf{Educators \& Language Tutors}\\
    Integrating the game into educational settings. 
    They are interested in using \textit{Spell Splash} as a supplementary tool to enhance student engagement in vocabulary learning. 
    Their feedback and insights will help refine the game’s content, ensuring alignment with educational goals and language acquisition strategies. 
    Additionally, their endorsement can increase the game's credibility and adoption in classrooms.

    \item \textbf{Parents \& Guardians}\\
    Supporters of young learners who use the game. 
    They are interested in ensuring that their children benefit from an effective and enjoyable educational tool. 
    Their concerns may include content appropriateness, safety features, and parental controls. 
    Their feedback can influence the game's development in terms of moderation, accessibility, and educational effectiveness.

    \item \textbf{Game Developers \& AI Engineers}\\
    Responsible for building and maintaining the game, ensuring that all AI-driven features and mechanics function as intended. 
    They are interested in creating a seamless, user-friendly, and engaging experience while incorporating adaptive learning mechanics, pronunciation recognition, and game balance. 
    Their expertise is crucial in continuously improving the game’s interactivity, efficiency, and learning outcomes.
\end{enumerate}

\section{User Stories}
\label{section:user-stories}

User stories capture the functional requirements of \textit{Spell Splash} from the perspective of different stakeholders. These simple, non-technical statements describe how users interact with the game to achieve their goals.

\vspace{1em}

\begin{longtable}{|>{\RaggedRight\arraybackslash\bfseries}p{4cm}|>{\RaggedRight\arraybackslash}p{11cm}|}
\hline
\textbf{Stakeholder} & \textbf{User Story} \\
\hline
\endfirsthead

\hline
\textbf{Stakeholder} & \textbf{User Story} \\
\hline
\endhead

\multirow{5}{=}{Student}
& As a student, I want to engage in an interactive RPG game so that I can improve my English vocabulary in a fun way. \\
\cline{2-2}
& As a student, I want to complete word-based challenges so that I can reinforce my understanding of word meanings and pronunciation. \\
\cline{2-2}
& As a student, I want to earn rewards and achievements so that I stay motivated to practice consistently. \\
\cline{2-2}
& As a student, I want to track my progress through analytics so that I can identify areas for improvement. \\
\cline{2-2}
& As a student, I want to adjust the difficulty level of vocabulary exercises so that I can learn at a pace that suits my proficiency. \\
\hline

\multirow{4}{=}{Educator \& Language Tutor}
& As an educator, I want to monitor my students’ progress so that I can provide targeted guidance and support. \\
\cline{2-2}
& As an educator, I want to recommend vocabulary challenges so that my students can focus on specific areas of improvement. \\
\cline{2-2}
& As an educator, I want to integrate Spell Splash into my curriculum so that my students can benefit from an engaging learning tool. \\
\cline{2-2}
& As an educator, I want to provide feedback on the game’s content so that it aligns with educational standards. \\
\hline

\multirow{3}{=}{Parent \& Guardian}
& As a parent, I want to ensure that the game is safe and appropriate for my child so that they can learn in a secure environment. \\
\cline{2-2}
& As a parent, I want to track my child's progress so that I can monitor their improvement in vocabulary learning. \\
\cline{2-2}
& As a parent, I want to set playtime limits so that my child balances gaming with other educational activities. \\
\hline

\multirow{4}{=}{Game Developer \& AI Engineer}
& As a developer, I want to implement AI-driven vocabulary exercises so that the game adapts to each player’s learning level. \\
\cline{2-2}
& As a developer, I want to ensure a smooth and bug-free gameplay experience so that users remain engaged. \\
\cline{2-2}
& As a developer, I want to integrate voice recognition features so that players can practice pronunciation effectively. \\
\cline{2-2}
& As a developer, I want to collect anonymized player data so that I can improve game mechanics based on real user behavior. \\
\hline

\end{longtable}

\section{Use Case Diagram}
\label{section:use-case-diagram}
<TIP: Write a use case diagram for your project here. Refer to an
article “What is a use case diagram?” by Lucidchart for help./>

\section{Use Case Model}
\label{section:use-case-model}
A use case is a detailed description of how a system
interacts with an external entity (such as a user or another system) to
accomplish a specific goal. Use cases provide a high-level view of the
functionality of a system and help in capturing and documenting its
requirements from the perspective of end users.

<TIP: Write use cases for your project here. Make sure to use the
appropriate type of use case for each scenario (brief, casual, and fully-dressed
use case)./>

\section{User Interface Design}
\label{section:user-interface-design}
<TIP: Put the initial design of your application here. You can
showcase a detailed design of a specific page or a sitemap of your application.
See an example below./>

\begin{figure}[h]
    \centering
    \includegraphics[width=0.8\textwidth]{examples/user-interface-design.png}
    \caption{User Interface Design}
\end{figure}